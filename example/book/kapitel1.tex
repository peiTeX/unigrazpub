\chapter{Kapitelüberschrift}

Bei Monografien entfallen die Zusammenfassungen je Kapitel. Die Kapitel sind nummeriert.

\section{Überschrift 1. Grades}

\blindtext

\subsection{Überschrift 2. Ebene – Auflistungen}

Innerhalb von Aufzählungen wird kein vertikaler Abstand gesetzt. Auch Absatzabstände entfallen hier.

\begin{itemize}
	\item Erster Eintrag
	\item Zweiter Eintrag
	\item Dritter Eintrag
	\begin{itemize}
		\item Erster Eintrag
		\item Zweiter Eintrag
		\item Dritter Eintrag
	\end{itemize}
\end{itemize}

\begin{enumerate}
	\item Erster Eintrag
	\item Zweiter Eintrag
	\item Dritter Eintrag
	\begin{enumerate}
		\item Erster Eintrag
		\item Zweiter Eintrag
		\item Dritter Eintrag
	\end{enumerate}
\end{enumerate}

\blindtext
\subsubsection{Wörtliche Zitate über mehrere Zeilen}

Kurze Wörtliche Zitate werden durch \enquote{Anführungszeichen} gekennzeichnet. Bei längeren wird der Text samt Quellenangabe in einer quote-Umgebung platziert.

\begin{quote}
	Dies ist ein wörtliches Zitat, das über mehrere Zeilen, mindestens jedoch drei
	Zeilen, geht und am Ende mit einer Fußnote belegt wird. Dies ist ein wörtliches
	Zitat, das über mehrere Zeilen, mindestens jedoch drei Zeilen, geht und am Ende
	mit einer Fußnote belegt wird.\autocite{test.2020}
\end{quote}

\subsubsection{Literaturverzeichnis}

Das Literaturverzeichnis wird automatisch mithilfe von bibatex und biber erzeugt. Hierfür wird der Stil biblatex-chicago geladen. 

