%% This is file `beitrag1.tex' version 1.00 (2022/10/05),
%% it is part of
%% UniGrazPub – LaTeX Templates for Graz University Library Publishing
%% ----------------------------------------------------------------------------
%%
%%  Copyright (C) 2022 by Marei Peischl <marei@peitex.de>
%%
%% ============================================================================
%% This work may be distributed and/or modified under the
%% conditions of the LaTeX Project Public License, either version 1.3c
%% of this license or (at your option) any later version.
%% The latest version of this license is in
%% http://www.latex-project.org/lppl.txt
%% and version 1.3c or later is part of all distributions of LaTeX
%% version 2008/05/04 or later.
%%
%% This work has the LPPL maintenance status `maintained'.
%%
%% The Current Maintainers of this work are
%%   Marei Peischl <unigrazpub@peitex.de>
%%
%% The development respository can be found at
%% https://github.com/peiTeX/unigrazpub
%% Please use the issue tracker for feedback!
%%
%% ============================================================================
%%
% !TeX program = lualatex
%%

\Article[
authorkeys={author1, author2},
subtitle=Untertitel,
doi=XXXX
]{Beitragstitel zweizeilig}


\begin{abstract}[ngerman]
	Dies ist eine Zusammenfassung. Zusammenfassungen bzw. Abstracts sollten ca.
	800 bis 900 Zeichen (inkl. Leerzeichen) umfassen. Dies ist eine Zusammenfassung.
	Zusammenfassungen bzw. Abstracts sollten ca. 800 bis 900 Zeichen (inkl.
	Leerzeichen) umfassen. Dies ist eine Zusammenfassung. Zusammenfassungen
	bzw. Abstracts sollten ca. 800 bis 900 Zeichen (inkl. Leerzeichen) umfassen. Dies
	ist eine Zusammenfassung. Zusammenfassungen bzw. Abstracts sollten ca. 800 bis
	900 Zeichen (inkl. Leerzeichen) umfassen. Dies ist eine Zusammenfassung.
	Zusammenfassungen bzw. Abstracts sollten ca. 800 bis 900 Zeichen (inkl.
	Leerzeichen) umfassen. Dies ist eine Zusammenfassung. Zusammenfassungen
	bzw. Abstracts sollten ca. 800 bis 900 Zeichen (inkl. Leerzeichen) umfassen. Dies
	ist eine Zusammenfassung. Zusammenfassungen bzw. Abstracts sollten ca. 800 bis
	900 Zeichen (inkl. Leerzeichen) umfassen.
	Schlagwörter:
	
	\keywords{Suchmaschine, treffend, drei, vier} 
\end{abstract}


\begin{abstract}[english]
	This is an abstract. Abstracts should contain approximately 800 to 900 characters
	(incl. spaces). This is an abstract. Abstracts should contain approximately 800 to
	900 characters (incl. spaces). This is an abstract. Abstracts should contain
	approximately 800 to 900 characters (incl. spaces). This is an abstract. Abstracts
	should contain approximately 800 to 900 characters (incl. spaces). This is an
	abstract. Abstracts should contain approximately 800 to 900 characters (incl.
	spaces). This is an abstract. Abstracts should contain approximately 800 to 900
	characters (incl. spaces). This is an abstract. Abstracts should contain
	approximately 800 to 900 characters (incl. spaces). This is an abstract. Abstracts
	should contain approximately 800 to 900 characters (incl. spaces). This is an
	abstract. Abstracts should contain approximately 800 to 900 characters (incl.
	spaces). 
	
	\keywords{search engine, matching, three, four}
\end{abstract}

%erzwungener Seitenumbruch vor Beginn des Textteils
\clearpage

\section{Überschrift 1. Grades}

Absatzumbrüche werden durch eine halbe Leerzeichen gekennzeichnet. Ein Absatzeinzug erfolgt nicht. 

\blindtext

\subsection{Überschrift 2. Ebene – Auflistungen}

Innerhalb von Aufzählungen wird kein vertikaler Abstand gesetzt. Auch Absatzabstände entfallen hier.

\begin{itemize}
	\item Erster Eintrag
	\item Zweiter Eintrag
	\item Dritter Eintrag
	\begin{itemize}
		\item Erster Eintrag
		\item Zweiter Eintrag
		\item Dritter Eintrag
	\end{itemize}
\end{itemize}

\begin{enumerate}
	\item Erster Eintrag
	\item Zweiter Eintrag
	\item Dritter Eintrag
	\begin{enumerate}
		\item Erster Eintrag
		\item Zweiter Eintrag
		\item Dritter Eintrag
	\end{enumerate}
\end{enumerate}

\blindtext
\subsubsection{Wörtliche Zitate über mehrere Zeilen}

Kurze Wörtliche Zitate werden durch \enquote{Anführungszeichen} gekennzeichnet. Bei längeren wird der Text samt Quellenangabe in einer quote-Umgebung platziert.

\begin{quote}
	Dies ist ein wörtliches Zitat, das über mehrere Zeilen, mindestens jedoch drei
	Zeilen, geht und am Ende mit einer Fußnote belegt wird. Dies ist ein wörtliches
	Zitat, das über mehrere Zeilen, mindestens jedoch drei Zeilen, geht und am Ende
	mit einer Fußnote belegt wird.\autocite{test.2020}
\end{quote}

\subsubsection{Abbildungen}
Abbildungen werden wie gewohnt eingefügt. Bei der Platzierung ist darauf zu achten, dass – sofern möglich – die Voreinstellung (htb) nicht verändert werden soll.
Abbildung \ref{fig:example-image} ist ein Beispiel für die Verwendung. 

\begin{figure}
	\centering
	\includegraphics[width=.5\linewidth]{example-image}
	\caption{Eine Beispielabbildung. Sie wird gleitend im Text platziert um den maximal möglichen Lesefluss zu erhalten. Um den Textbezug dadurch nicht zu verlieren wird auf die Abbildungsnummer verwiesen.}
	\label{fig:example-image}
\end{figure}


\subsubsection{Literaturverzeichnis}

Das Literaturverzeichnis wird automatisch mithilfe von BibLaTeX und biber erzeugt. Hierfür wird der Stil biblatex-chicago geladen. 

Die Quellen werden in einer bib-Datei je Beitrag gesammelt. Es ist Möglich durch die Herausgeber eine gesammelte Datei anzulegen um ggf. Datendopplungen zu vermeiden. Da jeder Beitrag jedoch ohnehin über ein eigenes Literaturverzeichnis verfügt ist dies nicht zwingend notwendig. 


\printbibliography